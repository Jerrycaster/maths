\documentclass[../research.tex]{subfile}

\subsection{Topological Manifolds}
\label{sec:3.1}

  \begin{prop}
    We say a topological space $X$ is \textnormal{second-countable} if the topology has a countable basis.

    $\R^n$ is second-countable.
  \end{prop}

  \begin{proof}
    The set of open balls
    $$S=\{B_r(q):r\in\mathbb{Q},q\in\mathbb{Q}^n\}$$
    forms a basis for the Euclidean topology, since any real open ball $B_\epsilon(x)$, where 
    $\epsilon\in\R$ and $x\in\R^n$, is a countable union of open balls $(B_{r_i}(q_i))_{i\in\N}\in 
    S$, where each $B_{r_i}(q_i)\subset B_\epsilon(x)$.
  \end{proof}

  \begin{definition}[Chart]
    Let $M$ be a second-countable topological space. A \textit{chart}, $(U,E,\varphi)$, is an ordered 
    tuple of open subsets $U\subset M,E\subset\R^n$ and a homeomorphism $\varphi:U\to E$.
  \end{definition}

  \begin{definition}[Atlas and manifold]
    An \textit{atlas} of $M$ is a collection of charts $\{(U_i,E_i,\varphi_i):i\in I\}$ such that the 
    $U_i$ cover $M$. I.e.,
    $$\bigcup_{i\in I}{U_i}=M.$$
    Then $M$ is said to be an \textit{$n$-manifold} which, colloquialy, is said to be locally homeomorphic 
    to $\R^n$.
  \end{definition}

  \begin{example}
    The 2-sphere, 
    $$S^2=\{x^2+y^2+z^2=r^2:x,y,z\in\R^3,r>0\},$$
    can be contructed as follows: \\
    
    We define an equivalence relation, $\sim$ , on the 2-disk, 
    $$D^2=\{x^2+y^2\leq r^2:x,y\in\R^2,r>0\},$$
    by $x\sim y\text{ if and only if }x^2+y^2=r^2.$
    Then the quotient space $D^2/\sim$ is homeomorphic to $S^2$.

    This amounts to ``gluing'' together the boundary points of $D^2$. $S^2$ is 2-manifold, that is, 
    locally homeomorphic to $\R^2$.
  \end{example}

  \begin{definition}[Transition map]
    Given two charts $(U_1,E_1,\varphi_1),(U_2,E_2,\varphi_2)$ such that $U_1\cap U_2\neq\varnothing$, 
    we define the \textit{transition map} between them by   
    $$\tau_{1,2}:\varphi_1(U_1\cap U_2)\to\varphi_2(U_1\cap U_2).$$

    Clearly $\tau_{1,2}$ is a homeomorphism, since $\tau_{1,2}=\varphi_2\circ\varphi_1^{-1}$. Hence 
    $\tau_{1,2}^{-1}=\varphi_1\circ\varphi_2^{-1}:=\tau_{2,1}$ is also a homeomorphism. Transition 
    maps give us a way of ``moving'' throughout an $n$-manifold in way that resembles 
    moving through $n$-dimensional Euclidean space.
  \end{definition}

\subsection{Differentiable Manifolds}
\label{sec:3.2}

  \begin{definition}[Differentiable manifold]
    We say a topological $n$-manifold $M$ is \textit{differentiable} if, for every pair of charts 
    $(U_i,E_i,\varphi_i),(U_j,E_j,\varphi_j)$ such that $U_i\cap U_j\neq\varnothing$, the transition 
    maps $\tau_{ij},\tau_{ji}$ are differentiable in $\R^n$. That is, all their partial derivatives 
    exist and are continuous.
    % We call $\tau_{ij}$ and $\tau_{ji}$ \textit{diffeomorphisms}, and say 
    % that $\varphi_i(U_i\cap U_j)$ and $\varphi_j(U_i\cap U_j)$ are \textit{diffeomorphic}.
  \end{definition}

  \begin{definition}[Differentiable function]
    Let $M$ be a differentiable $m$-manifold, $N$ a differentiable $n$-manifold, and $f:M\to N$ a continuous function. Fix charts 
    $(U_1,E_1,\varphi_1)$, $(U_2,E_2,\varphi_2)$ be charts for $M$, $N$ respectively. We say that
    $f$ is \textit{differentiable} if the function 
    $$\varphi_2\circ f\circ\varphi_1^{-1}:\R^m\to\R^n$$
    is differentiable.

    If $f$ is a homeomorphism, and $\varphi_1\circ f\circ\varphi_2^{-1}:\R^n\to\R^m$ differentiable, then 
    we say $f$ is a \textit{diffeomorphism}, and $M$ and $N$ are \textit{diffeomorphic}.

    Note this requires $m=n$, so that the Jacobian matrix of $\varphi_2\circ f\circ\varphi_1^{-1}$ is 
    invertible.
  \end{definition}

  \begin{example}
    If we let $N=\R^n$, then $f:M\to\R^n$ is differentiable if the function $f\circ\varphi^{-1}:\R^m
    \to\R^n$ is differentiable, for all charts $(U,E,\varphi)$ of $M$.
  \end{example}

  Given an $n$-dimensional differentiable manifold $M$, we can attach to every point of $M$ a 
  ``tangent space'' - a vector space isomorphic to $\R^n$.

  \begin{definition}[Tangent space]
    Fix $x\in M$, and choose any chart $(U,E,\varphi)$ with $x\in U$. Let $\Gamma$ be the set of 
    continuous and injective curves passing through $x$. That is,
    $$\Gamma=\{\gamma_i:(-1,1)\to M:i\in I\text{ s.t. }\gamma_i(0)=x\}.$$
    We then define an equivalence relation, $\sim$ , on $\Gamma$ by:
    $$\gamma_1\sim\gamma_2\text{ if and only if }(\varphi\circ\gamma_1)'(0)=(\varphi\circ\gamma_2)'(0).$$
    Then the \textit{tangent space}, $T_xM$, is equal to the quotient space $\Gamma/\sim$. We call 
    the equivalence classes in $T_xM$ \textit{tangent vectors}, which we denote $\gamma_i':=[\gamma_i]$.
  \end{definition}

  \textbf{Remark:} The definition of $\Gamma$ is somewhat arbitrary. More generally, we can define 
  $$\Gamma=\{\gamma_i:(a,b)\to M:i\in I\text{ s.t. }\gamma_i(c)=x\},$$
  where $a,b\in\R,a<b$, and $c\in(a,b)$. Then $\gamma_1\sim\gamma_2\text{ if and only if }
  (\varphi\circ\gamma_1)'(c)=(\varphi\circ\gamma_2)'(c),$ and we denote the tangent vectors $\gamma_i'(c)$.

  \begin{prop}
    If we define vector addition and scalar multiplication in $T_xM$ by 
    $$\gamma_1'+\lambda\gamma_2':=\{\gamma:(\varphi\circ\gamma)'(0)=(\varphi\circ\gamma_1)'(0)
    +\lambda(\varphi\circ\gamma_2)'(0),\text{ where }\gamma_1\in\gamma_1',\gamma_2\in\gamma_2',\lambda\in\R\},$$
    the identity element in $T_xM$ as the equivalence class of constant curves:
    \begin{align*}
      \gamma_0'&=\{\gamma:\gamma(t)=c\in U,\,\forall t\in (-1,1)\} \\
               &=\{\gamma:(\varphi\circ\gamma)'(0)=0\},
    \end{align*}
    and the inverse of $\gamma'$ by
    $$-\gamma':=\{-\gamma:(\varphi\circ(-\gamma))'(0)=-(\varphi\circ\gamma)'(0),\text{ where }\gamma\in\gamma'\}.$$
    Then $T_xM\cong\R^n$.
  \end{prop}

  \begin{definition}[Directional derivative]
    Given a differentiable $n$-manifold $M$ and a differentiable function $f:M\to\R^n$, we define 
    the \textit{directional derivative} of $f$ in the direction $\gamma'$ as 
    $$D_\gamma(f):=(f\circ\gamma)'(0),\,\forall\gamma\in\gamma'.$$
  \end{definition}

  \begin{definition}[Pushforward]
    If $\varphi:M\to N$ is a differentiable function of differentiable manifolds, then we define the 
    \textit{pushforward} of $\varphi$ at $x$ as the linear map 
    $$d\varphi_x:T_xM\to T_{\varphi(x)}N,$$
    defined by $d\varphi_x(\gamma')=(\varphi\circ\gamma)'(0)\,\,\forall\gamma\in\gamma'$.
  \end{definition}

  \subsubsection{Lie groups}
  \label{sec:3.2.1}

  \begin{definition}[Lie group]
  \end{definition}
