\documentclass[../summer.tex]{subfile}

\subsection{Sequences}
\label{sec:1.1}

  We denote set of positive integers as $\N$.

  \begin{prop}
    Let $a_i = (a_i^1,\ldots,a_i^n)\in\R^n$ be a sequence,  $\forall n\in\N$. Then $a_i\to a:=(a^1,\ldots,a^n)\in\R^n$, as $i\to\infty$ iff 
    $a_i^k\to a^k$ as $i\to\infty$, $\forall k\in\{1,\ldots,n\}$.
  \end{prop}

  \begin{proof}
    We know that 
    $$\max_{k=1,\ldots,n}{|a^k|}\leq\|a\|\leq\sqrt{n}\max_{k=1,\ldots,n}{|a^k|}.$$

    To show this, observe that:
    $$\sqrt{\max_k{((a^k)}^2)}\leq\sqrt{(a^1)^2+\ldots+(a^n)^2}\leq\sqrt{n\max_k{((a^k)^2})}$$
    $$\Rightarrow \sqrt{(\max_k{|a^k|})^2}\leq\|x\|\leq\sqrt{n(\max_k{|a^k|})^2}$$
    $$\Rightarrow \max_k{|a^k|}\leq\|x\|\leq\sqrt{n}\max_k{|a^k|}$$
    
    First assume that $a_i\to a$ as $i\to\infty$. Then $\forall\epsilon>0$ $\exists N\in\N$ such that 
    $\|a_i-a\|<\epsilon$ whenever $i\geq N$. \\

    Then $\forall k=1\ldots n$, $|a^k_i-a^k|\leq\max_k{|a^k_i-a^k|}\leq\|a_i-a\|<\epsilon$ whenever $i\geq N$, 
    so $a^k_i\to a^k$ as $i\to\infty$.\\

    Conversely, assume that $a_i^k\to a^k$ as $i\to\infty$, $\forall k\in\{1,\ldots,n\}$.
    Then $\forall\epsilon>0$, $\exists N^k\in\N$ such that $|a_i^k-a^k|\leq\frac{\epsilon}{\sqrt{n}}$ whenever $i\geq N^k$, $\forall k=1,\ldots,n$. Let 
    $$N=\max_{k=1,\ldots,n}{N^k}.$$

    Then $\max_k{|a^k_i-a^k|}<\frac{\epsilon}{\sqrt{n}}$ whenever $i\geq N$, so $\|a_i-a\|\leq\sqrt{n}\max_k{|a^k_i-a^k|}<\epsilon$.
  \end{proof}

  \begin{prop}
    The limit of a sequence in $\R^n$ is unique.
  \end{prop}

  \begin{proof}
    Let $(x_i)_{i\in\N}$ be a sequence with $x_i\in\R^n$. 
    Suppose for a contradiction that $x_i\to a$ and $x_i\to b$ as $i\to\infty$, with $a\neq b$.\\

    Then $\forall\epsilon>0$, $\exists N_a,N_b\in\N$ such that $\|x_i-a\|<\frac{\epsilon}{2}$ whenever 
    $i\geq N_a$, and $\|x_i-b\|<\frac{\epsilon}{2}$ whenever $i\geq N_b$. Let
    $$N=\max{\{N_a,N_b\}}.$$
    Then whenever $i\geq N$, we have:
    \begin{align*}
    \|a-b\|&=\|a-x_i+x_i-b\|\\
           &\leq\|a-x_i\|+\|x_i-b\|\\
           &<\frac{\epsilon}{2}+\frac{\epsilon}{2}=\epsilon,
    \end{align*}
    And since $\epsilon$ can be made arbitrarily small, we find that $a=b$, which is a contradiction.
  \end{proof}


  \begin{theorem}[Bolzano-Weierstrass]
    Let $(a_n)_{n\in\N}$ be a bounded sequence of real numbers. Then $(a_n)_n\in\N$ has a convergent subsequence.
  \end{theorem}

  \begin{proof}
    We say $x_i$ is a peak point of $(a_n)$ if $a_n\leq x_i, \forall n>i$.\\

    Then either we have infinitely many peak points, or finitely many.
    In the first case, we have an infinite subsequence, $(x_{i_j})_{j\in\mathbb{N}}$, of peak points which is non-increasing, 
    since $a_{i_n}\leq x_{i_j}$ whenever $n>j$. Then since $(a_i)$ is bounded, so is $(x_{i_j})$, so it is convergent.\\

    Now consider the case of finitely many peak points, say $x_1,\ldots x_k$. Choose 
    $n_1>\text{max}\{1,\ldots,k\}$. Since $a_{n_1}$ is not a peak point, $\exists n_2 > n_1$ such that 
    $a_{n_2}>a_{n_1}$. Since $a_{n_2}$ is not a peak point, $\exists n_3 > n_2$ such that 
    $a_{n_3}>a_{n_2}$. Continuing inductively, with $a_{n_i}>a_{n_{i-1}}$, we obtain a decreasing subsequence 
    $(a_{n_j})_{j\in\N}$. Then since $(a_{n_j})_{j\in\N}$ is bounded, it converges.
  \end{proof}

  \begin{cor}
    Let $(a_i)_{i\in\N}$ be a bounded sequence, with $a_i\in\R^n\forall n\in\N$. Then $(a_i)$ has a convergent subsequence.
  \end{cor}
  \begin{proof}
    Since $(a_i)$ is bounded, and the component sequences $(a_i^k)$ are bounded $\forall k=1,\dots,n$.\\

    Then by Bolzano-Weierstrass, $(a_i^1),\dots,(a_i^n)$ have subsequences $(a_{i_{j_1}}^1), \dots, (a_{i_{j_n}}^n)$ 
    converging to $a^1,\dots,a^n$ respectively. Let 
    \begin{align*}
      J_1&=\{j_{1_1},j_{1_2},\dots\},\\
      J_2&=\{j_{2_1},j_{2_2},\dots\},\\
         &\vdots \\
      J_n&=\{j_{n_1},j_{n_2},\dots\},
    \end{align*}
    and $J=J_1\cap\dots\cap J_n$.\\

    Then each component subsequence $(a_j^k)_{j\in J}$ converges to $a^k$, so $(a_j)_{j\in J}$ converges to $a$.
  \end{proof}

\subsection{Continuous Functions}
\label{sec:1.2}

  \begin{definition}[Continuity]
    Let $A\subset\R^n$ be open. We say $f:A\to\R^m$ is \textit{continuous} on $A$ if, 
    $\forall a\in A$, $\forall\epsilon>0$ $\exists\delta>0$ such that 
    $\|f(x)-f(a)\|<\epsilon$, whenever $\|x-a\|<\delta$.
  \end{definition}

  \begin{prop}
    For $f:A\to\R^m$, we write $f(x)=(f_1(x),\dots,f_m(x))$.\\

    Then $f$ is continuous if and only if each of the component functions 
    $f_i:A\to\R$ are continuous.
  \end{prop}

  \begin{proof}
    First assume $f$ is continuous at $a\in A$. Then $\forall\epsilon>0$ 
    $\exists\delta>0$ such that $\|f(x)-f(a)\|<\epsilon$ whenever $\|x-a\|<\delta$. Then 
    \begin{align*}
      |f_i(x)-f_i(a)|&\leq\max_i{|f_i(x)-f_i(a)|}\\
                     &\leq\|f(x)-f(a)\|\\
                     &<\epsilon\quad\forall i=1,\dots,m.
    \end{align*}

    Hence each $f_i$ is continuous at $a$. Now assume that each $f_i$ is continuous at $a\in A$.\\

    Then $\forall\epsilon>0$, $\exists\delta_i>0$ such that $|f_i(x)-f_i(a)|<\frac{\epsilon}{m\sqrt{m}}$
    whenever $\|x-a\|<\delta_i$, $\forall i=1,\dots,m$. Now let $\delta=\min{\{\delta_1,\dots,\delta_m\}}$\\

    Then when $\|x-a\|<\delta$, $\|x-a\|<\delta_i$, so $|f_i(x)-f_i(a)|<\frac{\epsilon}{m\sqrt{m}}$ 
    $\forall i=1,\dots,m$. Then
    \begin{align*}
    \|f_i(x)-f_i(a)\|&\leq\sqrt{m}\max_i{|f_i(x)-f_i(a)|}\\
                     &\leq\sqrt{m}\sum_{i=1}^m{|f_i(x)-f_i(a)|}\\
                     &<\epsilon
    \end{align*}    
  \end{proof}

  \begin{example}
    Every linear function is continuous.
  \end{example}

  \begin{prop}
    Let $A\subset\R^n$ be open. Then $f:A\to\R^m$ is continuous at $x\in A$ if and only if 
    $f(x_i)\to f(x)$ as $i\to\infty$ for any sequence $(x_i)$ converging to $x$.
  \end{prop}

  \begin{proof}
    Let $(x_i)$ be a sequence in $A$ converging to $x\in A$, and assume $f:A\to\R^m$ to be continuous at $x$.
    Then $\forall\epsilon>0$, $\exists\delta>0$ such that $\|f(x)-f(y)\|<\epsilon$ 
    whenever $0<\|x-y\|<\delta$.\\

    Since $x_i\to x$, $\nat$ such that $\|x_i-x\|<\delta$ whenever $i\geq N$, and hence $\|f(x_i)-f(x)\|<\epsilon$, 
    so we have that $f(x_i)\to f(x)$ as $i\to\infty$.\\

    Now assume the converse holds, and for a contradiction that $f$ is not continuous. Then $\exists\epsilon>0$ 
    such that $\forall\delta>0$ $\|f(x)-f(x_i)\|\geq\epsilon$ whenever $0<\|x-x_i\|<\delta$, 
    and in particular whenever $0<\|x-x_i\|<\frac{1}{i}$, $\forall i\in\N$.\\

    Then the sequence $(x_i)$ converges to $x$, so by assumption the image sequence $(f(x_i))$ converges to $f(x)$.\\

    Thus $\exists N\in\N$ such that $\|f(x_i)-f(x)\|<\epsilon$ whenever $i\geq N$, which is a contradiction.
    Hence, $f$ is continuous at $x$.
  \end{proof}

  \begin{theorem}[Extreme Value Theorem]
    Let $E\subset\R^n$ be compact and non-empty. Let $f:E\to\R$ be continuous.
    Then $f$ achieves its maximum and minimum values on $E$.
  \end{theorem}

  \begin{proof}
    We prove that $f$ reaches its maximum on $E$. First suppose for a contradiction that 
    $f$ is unbounded above on $E$. Then $\forall M\geq 0$, and in particular 
    $\forall i\in\N$, $\exists x_i\in E$ such that $f(x_i)>i$. 
    In other words, $f(x_i)\to\infty$ as $i\to\infty$.\\

    Then since $E$ is compact, it is bounded, so the sequence $(x_i)_{i\in\N}$ is bounded. 
    Then by the Bolzano-Weierstrass theorem it has a subsequence $(x_{i_j})_{j\in\N}$ convergent to $x\in E$.\\

    Now since $f$ is continuous on $E$, it is continuous at $x$, so $f(x_{i_j})\to f(x)$ as $j\to\infty$.
    Since $E$ is compact, it is closed, so $f(x)\in E$, and therefore $f(x)$ is finite. This is a contraction since 
    $f$ was assumed to diverge to infinity for any subsequence of $(x_i)$, including $(x_{i_j})$.\\

    So $f$ is bounded above on $E$. Now by completeness of $\R$, $f(E)$ has a supremum, say 
    $$M=\sup_{x\in E}{f(x)}.$$ 
    We show that that $f$ reaches its maximum on $E$, i.e. that $M\in f(E)$, or 
    that $\exists p_+\in E$ such that $f(p_+)=M$. Now $\forall i\in\N$, $\exists x_i\in E$ such that 
    $$M-\frac{1}{i}<f(x_i)\leq M.$$
    Then clearly $f(x_i)\to M$ as $i\to\infty$, so by the Bolzano-Weierstrass theorem $(x_i)_{i\in\N}$ has 
    a subsequence $(x_{i_j})_{j\in\N}$ convergent in to some $p_+\in E$. Since $f$ is continuous at $p_+$, 
    $f(x_{i_j})\to f(p_+)$ as $j\to\infty$. Hence $f(p_+)=M$, so $f$ achieves its maximum on $E$.\\

    Now in a similar fashion we prove that $f$ reaches its minimum on $E$. Suppose that 
    $f$ is unbounded below on $E$. Then $\forall i\in\N$, $\exists x_i\in E$ such that $f(x_i)<-i$. 
    So $f(x_i)\to-\infty$ as $i\to\infty$.\\

    Now, the sequence $(x_i)_{i\in\N}$ is bounded. Then by the Bolzano-Weierstrass theorem it has a 
    subsequence $(x_{i_j})_{j\in\N}$ convergent to $x\in E$. Since $f$ is continuous at $x$, $f(x_{i_j})\to f(x)$ as $j\to\infty$, 
    and so $f(x)\in E$ (so $f(x)$ is finite). This is a contraction since 
    $f$ was assumed to diverge to negative infinity for any subsequence of $(x_i)$, including $(x_{i_j})$.\\

    So $f$ is bounded below on $E$. Now by completeness of $\R$, $f(E)$ has an infimum, say 
    $$m=\inf_{x\in E}{f(x)}.$$ 
    We show that that $f$ reaches its minimum on $E$, i.e. that $m\in f(E)$, or 
    that $\exists p_-\in E$ such that $f(p_-)=m$. Now $\forall i\in\N$, $\exists x_i\in E$ such that 
    $$m\leq f(x_i)<m+\frac{1}{i}.$$
    Then clearly $f(x_i)\to m$ as $i\to\infty$, so by the Bolzano-Weierstrass theorem $(x_i)_{i\in\N}$ has 
    a subsequence $(x_{i_j})_{j\in\N}$ convergent in to some $p_-\in E$. Since $f$ is continuous at $p_-$, 
    $f(x_{i_j})\to f(p_-)$ as $j\to\infty$. Hence $f(p_-)=m$, so $f$ achieves its minimum on $E$.
  \end{proof}

\subsection{Uniform Continuity and Convergence}
\label{sec:1.3}

  \begin{definition}[Uniformly Continuous]
    Let $A\subset\R^n$ be open. We say $f:A\to\R^m$ is \textit{uniformly continuous} on $A$ if, 
    $\forall\epsilon>0$ $\exists\delta>0$ such that $\forall x,y\in A$,
    $\|f(x)-f(y)\|<\epsilon$ whenever $\|x-y\|<\delta$.
  \end{definition}

  \textbf{Remark:} Clearly uniform continuity implies continuity.

  \begin{prop}
    Let $f:\R\to\R$ be differentiable with bounded derivative. Then $f$ is uniformly continuous.
  \end{prop}

  \begin{proof}
    Fix $\eps$. Then $\forall x,y\in\R$ with $x<y$, by the Mean Value Theorem 
    $\exists c\in(x,y)$ such that $f'(c)=\frac{f(y)-f(x)}{y-x}$. By assumption 
    $\exists M\geq 0$ such that $|f'(c)|=\frac{|f(y)-f(x)|}{|y-x|}\leq M$.\\

    Set $\delta = \frac{\epsilon}{M+1}$. Then 
    \begin{align*}
      |f(y)-f(x)|&\leq M|y-x|\\
                 &<\frac{M\epsilon}{M+1}\\
                 &<\frac{M\epsilon}{M}\\
                 &=\epsilon
    \end{align*}
  \end{proof}

  \begin{prop}
    A continuous function on a compact subset $E\subset\R^n$ is uniformly continuous.
  \end{prop}

  \begin{proof}
    Let $E\subset\R^n$ be compact, and $f:E\to\R^m$ continuous. We assume for a contradiction that 
    $f$ is not uniformly continuous. Then $\exists\epsilon>0$ such that $\forall i\in\N$ 
    $\exists x_i,y_i\in E$ with $\|x_i-y_i\|<\frac{1}{i}$ such that $\|f(x_i)-f(y_i)\|\geq\epsilon$.\\

    Then by the Bolzano-Weierstrass theorem, the sequence $(x_i)_{i\in\N}$ has a
    subsequences $(x_{i_j})_{j\in\N}$ converging to $x\in E$.  Furthermore, the sequence $(y_i)_{i\in\N}$ 
    has a convergent subsequence $(y_{i_j})_{j\in\N}$. We now show that $y_{i_j}\to x$ as $j\to\infty$.

    Choose $\epsilon'>0$. Then $\exists N\in\N$ such that $\|x_{i_j}-x\|<\frac{\epsilon'}{2}$ whenever $i_j\geq N$.
    \begin{align*}
    \|x-y_{i_j}\|&=\|x-x_{i_j}+x_{i_j}-y_{i_j}\|\\
                 &\leq\|x-x_{i_j}\|+\|x_{i_j}-y_{i_j}\|\\
                 &<\frac{\epsilon'}{2}+\frac{1}{i_j}\\
                 &\leq\frac{\epsilon'}{2}+\frac{1}{j}\\
                 &<\frac{\epsilon'}{2}+\frac{\epsilon'}{2}\\
                 &=\epsilon'
    \end{align*}
    
  \end{proof}
