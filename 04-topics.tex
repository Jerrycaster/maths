\documentclass[../research.tex]{subfile}

\subsection{Lebesgue Integration}
\label{sec:4.1}

\subsection{The Banach-Tarski Paradox}
\label{sec:4.3}

\subsection{Order Theory}
\label{sec:4.4}

    \begin{definition}[Partial order]
        Given a set $P$, a \textit{partial order}, $\leq$, is a binary relation on $P$ satisying the 
        following:
        \begin{enumerate}
            \item Reflexivity: $p\leq p \quad\forall p\in P$
            \item Antisymmetry: $p\leq q\wedge q\leq p\Rightarrow$ $p=q\quad\forall p,q\in P$
            \item Transitivity: $p\leq q$, $q\leq r$ $\Rightarrow$ $p\leq r\quad\forall p,q,r\in P$
        \end{enumerate}
        We call a set with a partial order a \textit{poset}, $(P,\leq)$.
    \end{definition}

    \begin{definition}[Lattice]
    \end{definition}

\subsection{Diffeology}
\label{sec:4.5}

\subsection{Tensors}
\label{sec:4.6}

    \subsubsection{Dual space of a vector space}
    \label{sec:4.6.1}

        \begin{definition}[Linear functional]
        \end{definition}

    \subsubsection{Cotangent vectors}
    \label{sec:4.6.2}

    \subsubsection{Differential Forms}
    \label{sec:4.6.3}


