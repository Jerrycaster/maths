\documentclass[../summer.tex]{subfile}

\subsection{Quotients}
\label{sec:2.1}

  \begin{definition}[Binary relation]
    Let $X$ be a non-empty set. A \textit{binary relation}, $\sim$, on $X$ is a map 
  $$\cdot\sim\cdot:X\times X\to\{\verb%True, False%\}.$$
  \end{definition}

  \begin{definition}[Equivalence relation]
    Let $\sim$ be a binary operation on a non-empty set $X$. We say $\sim$ is an 
    \textit{equivalence relation} if it is
    \begin{enumerate}
      \item Reflexive: $x\sim x \quad\forall x\in X$
      \item Symmetric: $x\sim y$ $\Leftrightarrow$ $y\sim x\quad\forall x,y\in X$
      \item Transitive: $x\sim y$, $y\sim z$ $\Rightarrow$ $x\sim z\quad\forall x,y,z\in X$
    \end{enumerate}
  \end{definition}

  \begin{prop}
    Let $\sim$ be an equivalence operation on a non-empty set $X$. Given an $x\in X$, we define 
    the \textit{equivalence class} of $x$, $[x]$, to be the set of elements in $X$ equivalent to $x$ 
    under $\sim$. In other words:
    $$[x]=\{y\in X:y\sim x\}.$$
    Then the equivalence classes partition $X$.
  \end{prop}

  \begin{proof}
    Since every element $x_i\in X$, $i\in I$, is equivalent to itself, $x_i\in[x_i]$, so 
    $$X = \bigcup_{i\in I}{[x_i]}.$$
    So equivalence classes cover $X$.\\

    Then if $x\nsim y$, $y\nsim x$, so $y\notin[x]$. And if $z\in[y]$, $z\sim y$ so $z\nsim x$, 
    and so $z\notin [x]$. Hence
    $$x\nsim y \Leftrightarrow [x]\cap[y]=\varnothing.$$
    So equivalence classes partition $X$.
  \end{proof}

  \begin{definition}[Quotient set]
    Let $X$ be a non-empty set, and $\sim$ an equivalence relation on $X$. We define the \textit{quotient set}, or \textit{quotient space}, 
    $\X$ to be the set of equivalence classes of $X$. I.e.
    $$\X=\{[x]:x\in X\}.$$
  \end{definition}

  \begin{definition}[Quotient group]
    Let $(G,+)$ be a group, and $N$ a normal subgroup of $G$. We define an equivalence relation 
    $\sim$ on $G$ to be such that 
    $$x\sim y\text{ if and only if }N+x=N+y.$$    
    Then equivalence classes $[x]$ are the cosets of $N$ in $G$, $N+x=\{n+x:n\in N\}$, so the quotient set
    $$\G=\{N+x:x\in G\}.$$
    Equivalently, two elements of $G$ are equivalent if they lie in the same coset. 
    Then we define the \textit{quotient group} 
    $$\frac{G}{N}=(\G,+)$$
    where the binary operation $\cdot+\cdot:\frac{G}{N}\times\frac{G}{N}\to\frac{G}{N}$ is defined by 
    $$(N+x)+(N+y)=N+(x+y).$$
  \end{definition}

  \begin{prop}
    $\frac{G}{N}$ is indeed a group under $+$.
  \end{prop}

  \begin{proof}
    We first check that $+$ is well-defined on $\frac{G}{N}$. \\
    
    Suppose that $N+x=N+x'$, and $N+y=N+y'$. 
    So $x-x'\in N$ and $y-y'\in N$. Then $(x-x')+(y-y')\in N$ this implies that $(x+y)-(x'+y')\in N$, 
    since $N$ is a normal subgroup.\\

    Then $N+(x+y)=N+(x'+y')$, which implies that $(N+x)+(N+y)=(N+x')+(N+y')$.\\

    So $+$ is well-defined. We now check the group axioms:
    \begin{enumerate}
      \item \textit{Closure}: Since $G$ is closed under $+$, $\frac{G}{N}$ is clearly closed under $+$.
      \item \textit{Associativity}: \begin{align*}
                                      \forall x,y,z\in G, (N+x)+((N+y)+(N+z)) &= (N+x)+(N+(y+z)) \\
                                                                              &= N+(x+(y+z)) \\
                                                                              &= N+((x+y)+z) \\
                                                                              &= (N+(x+y))+(N+z) \\
                                                                              &= ((N+x)+(N+y))+(N+z)
                                    \end{align*}
      \item \textit{Identity}: $N$ is the identity. To see this, observe that $N+(N+x)=N+x=(N+x)+N$.
      \item \textit{Inverses}: Given $N+x$, $-(N+x)=N+(-x)$. To see this, observe that $(N+x)+(N+(-x))=N=(N+(-x))+(N+x)$.
    \end{enumerate}
  \end{proof}

  \begin{definition}[Quotient topology]
    Let $(X,\tau_X)$ be a topological space. We define the \textit{natural map}, $p:X\to \X$, by $p(x)=[x]$ $\forall x\in X$. 
    We define the \textit{quotient topology}, $\tau_\sim$, to be
    $$\tau_\sim = \{U\in \X:p^{-1}(U)\in\tau_X\}.$$
    Note that $\tau_\sim$ is finest topology on $\X$ such that the natural map $p$ is continuous.
    We say that it is the \textit{final topology} on $\X$ with respect to $p$.
  \end{definition}

  \begin{definition}[Quotient map]
    We say $q:X\to Y$ between topological spaces $X$ and $Y$ is a \textit{quotient map} 
    if it is surjective, and $U\subset Y$ is open in $Y$ if and only if $q^{-1}(U)\subset X$ 
    is open in $X$.
  \end{definition}

  \begin{example}
    The natural map $p:X\to \X$ is a quotient map.
  \end{example}

  \begin{proof}
    $p$ is surjective, since $\forall[x]\in \X$, $p(x)=[x]$. The second property follows from the 
    definition of $p$.
  \end{proof}

  \begin{prop}
    Let $q:X\to Y$ be a quotient map. Then $f:Y\to Z$ is continuous if and only if 
    $f\circ q:X\to Z$ is continuous.
  \end{prop}

  \begin{proof}
    If $f$ is continuous, then $f\circ q$ is continuous by composition. Now assume 
    $f\circ q$ to be continuous, and let $U\subset Z$ be open in $Z$. Then 
    $(f\circ q)^{-1}(U)=q^{-1}(f^{-1}(U))$ is open in $X$, so $f^{-1}(U)$ is open in $Y$.
  \end{proof}

  \begin{prop}[Universal property of quotients]
    If $f:X\to Y$ is continuous and constant on equivalence classes of $X$. I.e. 
    $f|_{[x]}$ is constant $\forall[x]\in\X.$\\

    Then there exists a unique continuous map $g:\X\to Y$ such that 
    $$f=g\circ p.$$
    We say that $f$ descends to the quotient.
  \end{prop}

  \begin{proof}
    So given $x_i\in X$, $f(x)=c_i$ $\forall x\in[x_i]$. Then we define $g:\X\to Y$ by 
    $g([x_i])=c_i$. Clearly this choice of $g$ is unique. We first check this is well-defined:\\

    Suppose $[x_i]=[y_i]$. Then if $y\in[y_i]$, $y\in[x_i]$, so $g([y_i])=c_i$. Now, 
    $g\circ p(x_i)=g(p(x_i))=g([x_i])=c_i=f(x_i)$. Hence $g\circ p=f$.\\

    We now check $g$ is continuous. Since $p$ is a quotient map and $g\circ p$ is continuous, 
    $g$ is continuous by the previous proposition.
  \end{proof}

\subsection{Group Actions}
\label{sec:2.2}

  \begin{definition}[Automorphism group]
    Given an object $X$, an \textit{automorphism} is an isomorphism, $\phi$, from $X$ to itself. Explicitly,
    $$\phi:X\to X.$$
    Coloquially, automorphisms are invertible mappings that preserve structure. In fact, they form a group under composition, 
    called the \textit{automorphism group} of $X$, which we denote $\text{Aut}(X)$.
  \end{definition}

  \begin{example}
    \begin{verbatim}
    \end{verbatim}
    \begin{enumerate}
      \item If $X$ is a set with $n$ elements, then
            $$\text{Aut}(X)=S_n,$$
            the group of permutations of the elements of $X$.
      \item If $X$ is an $n$-dimensional vector space over a field $F$, then 
            $$\text{Aut}(X)=\text{GL}_n(F),$$
            the group of invertible $n\times n$ matrices over $F$, or equivalently the group of invertible linear 
            transformations from $X$ to itself.
    \end{enumerate}
  \end{example}

  We can think of $\text{Aut}(X)$ as \textit{acting} on $X$ in an invertible way. In particular,

  \begin{definition}[Group action]
    Let $G$ be a group and $X$ a set. The \textit{(left) group action}, $\phi$, is a function
    $$\phi:G\times X\to X.$$
    such that the following hold $\forall x\in X$, where we denote $\phi(g,x)$ by $g\cdot x$, and $e$ the identity in $G$:
    \begin{enumerate}
      \item \textit{Identity}: $e\cdot x=x$
      \item \textit{Compatibility}: $(gh)\cdot x=g\cdot(h\cdot x)\quad\forall g,h\in G$
    \end{enumerate}
    Here $G$ is called a \textit{transformation group}, $X$ is called a \textit{$G$-set}, and 
    $\phi$ is a \textit{group action}.
  \end{definition}

  We can think of a group action as a homomorphism $\psi:G\to\text{Aut}(X):g\mapsto\sigma$, where $\sigma(x)=g\cdot x$. Then the action of $g\in G$ 
  on $x\in X$ is equivalent to applying $\sigma$ to $x$.

  \begin{definition}[Orbit]
    Consider a group $G$ acting on a set $X$. We define the \textit{orbit}, $G\cdot x$,
    of $x\in X$ to be the set of elements in $X$ to which $x$ can be mapped to by the elements of $G$. Explicitly:
    $$G\cdot x=\{g\cdot x:g\in G\}.$$
  \end{definition}

  We can define an equivalence relation $\sim$ on $X$ by
  $$x\sim y\text{ if and only if }G\cdot x=G\cdot y,\quad x,y\in X.$$
  Equivalently, $x\sim y$ if and only if $\exists g\in G$ such that $g\cdot x=y$ (and so $g^{-1}\cdot y=x$). 
  Then the orbits of $X$, under the action of $G$, partition $X$. The quotient set induced by $\sim$ is
  $$\X=\{G\cdot x:x\in X\}.$$
  This is the set of all orbits of $X$ under the action of $G$, called the \textit{quotient} of the action and denoted $\frac{X}{G}$.

  \begin{definition}[Stabiliser subgroup]
    Given $g\in G$ and $x\in X$ such that $g\cdot x=x$, we say that $g$ fixes $x$, in other words 
    $g$ acts as the identity on $x$. So for all $x\in X$, we define the \textit{stabiliser subgroup}, $G_x\subset G$, of $G$ with respect to 
    $x$ as the set of elements in $G$ that fix $x$. Explicitly, given $x$,
    $$G_x=\{g\in G:g\cdot x=x\}.$$
  \end{definition}

  \begin{prop}
    $G_x$ is indeed a subgroup of $G$.
  \end{prop}

  \begin{prop}
    Let $\psi:G\to\text{Aut}(X)$ be a homomorphism. Then
    $$\text{Ker}(\psi)=\bigcap_{x\in X}{G_x}.$$
  \end{prop}

  $G_x$ is generally not a normal subgroup of $G$. However,

  \begin{lemma}
    Given $x,y\in X$, and $g\in G$ such that $y=g\cdot x$, 
    the two stabiliser groups $G_x,G_y$ are related by 
    $$G_y=g^{-1}G_x g.$$
  \end{lemma}

  \begin{proof}
    Fix $x,y\in X$, and let $g\in G$ be such that $y=g\cdot x$, and fix $h\in G_y$. Then 
    \begin{align*}
      h\cdot y= y &\Longleftrightarrow h\cdot(g\cdot x)=g\cdot x \\
                  &\Longleftrightarrow (hg)\cdot x=g\cdot x \\
                  &\Longleftrightarrow g^{-1}\cdot((hg)\cdot x)=g^{-1}\cdot(g\cdot x) \\
                  &\Longleftrightarrow (g^{-1}hg)\cdot x=(g^{-1}g)\cdot x = x \\
                  &\Longleftrightarrow g^{-1}hg\in G_x \\
                  &\Longleftrightarrow h\in g^{-1}G_xg
    \end{align*}
  \end{proof}

  \begin{theorem}[Orbit-stabiliser theorem]
    For a fixed $x\in X$, there exists a bijection $\psi:\frac{G}{G_x}\to G\cdot x.$
  \end{theorem}

  \begin{proof}
    We define a map $\psi:\frac{G}{G_x}\to G\cdot x$ by $\psi(G_xg)=g\cdot x$, $\forall g\in G$.
    First, we show this an injective, well-defined function: Fix $x\in X$, and let $G_xg=G_xh$. Then 
    \begin{align*}
      G_xg^{-1}h=G_x &\Longleftrightarrow g^{-1}h\in G_x \\
                     &\Longleftrightarrow (g^{-1}h)\cdot x=g^{-1}\cdot(h\cdot x)=x \\
                     &\Longleftrightarrow h\cdot x=g\cdot x \\
                     &\Longleftrightarrow \psi(G_xg)=\psi(G_xh)
    \end{align*}
    Now we show that $\psi$ is surjective: Fix $x\in X$, and $g\cdot x\in G\cdot x$.
    $$G_xg=\{hg:h\cdot x=x, \,\forall h\in G\}$$
  \end{proof}

  \begin{cor}[Burnside's lemma]
    The number of orbits of $G$ in $X$, $|X/G|$, is given by
    $$|X/G|=\frac{1}{|G|}\sum_{g\in G}{|X^g|},$$
    where $X^g\subset X$ is the set of elements in $X$ fixed by $g$.
  \end{cor}  
  
  In fact, if $G$ is finite, then by Lagrange's theorem,
  $$|G|=|G\cdot x||G_x|.$$

  \begin{definition}[Topological group]
    A \textit{topological group} is a non-empty set $G$ which is both a group and topological space, 
    in which the group operations of product:
    $$G\times G\to G:(x,y)\mapsto xy,\quad x,y\in G$$
    and taking inverses:
    $$G\to G:x\mapsto x^{-1},\quad x\in G$$
    are continuous, where $G\times G$ is given the product topology.
  \end{definition}

  \begin{definition}[Homomorphism of topological groups]
    Let $G,H$ be topological groups. Then a \textit{homomorphism of topological groups} is a 
    structure-preserving map (i.e. a group homomorphism), $\phi:G\to H$, which is continuous. Explicitly,
    \begin{enumerate}
      \item $\phi(xy)=\phi(x)\phi(y),\quad x,y\in G$
      \item $\phi^{-1}(U)$ is open in $G$ whenever $U$ is an open subset of $H$.
    \end{enumerate}

    An \textit{isomorphism} is a bijective group homomorphism which is also a homeomorphism of the 
    underlying topological spaces.
  \end{definition}

  \begin{definition}[Continuous group action]
  \end{definition}

\subsection{Initial and Final Topologies}
\label{sec:2.3}

  Given a set $X$, the indiscrete topology is the coursest topology we can give to $X$. 
  However, for a family of topological spaces $Y_i$ indexed by $i\in I$, the functions 
  $$f_i:X\to Y_i$$
  will almost never be continuous. The initial topology is the coursest topology such that 
  they are.

  \begin{definition}[Initial topology]
    Given a family of topological spaces $Y_i$ indexed by $i\in I$, the \textit{initial topology} 
    with respect to the functions
    $$f_i:X\to Y_i$$
    is the coursest topology that can be given to $X$ such that those functions are continuous.\\

    Explicitly, it is the topology generated by taking all finite intersections and arbitrary unions of 
    sets of the form $f_i^{-1}(U_i)$, where $U_i$ is open in $Y_i$. Consider the following examples:

    \begin{enumerate}
      \item Let $A\subset X$. The subspace topology $\tau_A=\{A\cap U:U\text{ is open in }X\}$
            is the initial topology on $A$ with respect to the canonical inclusion map 
            $$i:A\hookrightarrow X.$$
      \item Given a family of topological spaces $X_i$ indexed by $I$, let $X=\prod_{i\in I}{X_i}.$
            the product topology on $X$ is the initial topology on $X$ with respect to 
            the canonical projections 
            $$p_i:X\to X_i\quad\forall i\in I.$$
    \end{enumerate}
  \end{definition}

  \begin{theorem}[Characteristic property of initial topologies]
    Let $X,Y,Z_i$ be topological spaces, with $Z_i$ indexed by $i\in I$ and $Y$ given the initial 
    topology with respect to the functions
    $$g_i:Y\to Z_i.$$

    Let $f:X\to Y$. Then $f$ is continuous if and only if $g_i\circ f$ is continuous $\forall i\in I$.\\

    In other words, the following diagram is commutative for all $i\in I$:\\
    \begin{center}
      \begin{tikzcd}
        X \arrow{r}{f} \arrow[swap]{dr}{g_i\circ f} & Y \arrow{d}{g_i}\\
         & Z_i
      \end{tikzcd}
    \end{center}
  \end{theorem}
  
  \begin{proof}
    Assume first that $f$ is continuous. Then since $g_i$ is continuous $\forall i\in I$, $g_i\circ f$ is continuous.\\

    Now assume $g_i\circ f$ is continuous $\forall i\in I$. Let $U\subset Y$ be open in $Y$. Then $U$ is a finite intersection 
    and arbitrary union of open sets of the form $g_i^{-1}(V_i)$, where each $V_i$ is open in $Z_i$.\\

    Then $f^{-1}(U)$ is a finite intersection and arbitrary union of open sets of the form $f^{-1}(g_i^{-1}(V_i))=
    (g_i\circ f)^{-1}(V_i)$. This is equal to $(g_i\circ f)^{-1}(W_i)$, where $W_i$ is a finite intersection and arbitrary union 
    of the open sets $V_i$.\\

    Then since each $W_i$ is open, and each $g_i\circ f$ is continuous, $(g_i\circ f)^{-1}(W_i)=f^{-1}(g_i^{-1}(W_i))$ is open in $X$.
    $g_i^{-1}(W_i)$ is a finite intersection and arbitrary union of $g_i^{-1}(V_i)$, which is open in $Y$. Hence $f^{-1}(U)$ is open in $X$.
  \end{proof}

  \begin{definition}[Final topology]
    Given a family of topological spaces $X_i$ indexed by $i\in I$, the \textit{final topology} 
    with respect to the functions
    $$f_i:X_i\to Y$$
    is the finest topology that can be given to $Y$ such that those functions are continuous.\\

    Explicitly, a subset $U\subset Y$ is an element of the final topology if and only if 
    $f_i^{-1}(U)$ is open in $X_i$, $\forall i\in I$.
  \end{definition}

  \begin{theorem}[Universal property of final topologies]
    Let $X_i,Y,Z$ be topological spaces, with $X_i$ indexed by $i\in I$ and $Y$ given the final 
    topology with respect to the functions
    $$f_i:X_i\to Y$$

    Then $g:Y\to Z$ is continuous if and only if $g\circ f_i$ is continuous $\forall i\in I$.\\

    In other words, the following diagram is commutative $\forall i\in I$:\\
    \begin{center}
      \begin{tikzcd}
        X_i \arrow{r}{f_i} \arrow[swap]{dr}{g\circ f_i} & Y \arrow{d}{g}\\
         & Z
      \end{tikzcd}
    \end{center}
  \end{theorem}

\subsection{Review of point-set topology}
\label{sec:2.4}

  \begin{prop}
    Any compact subset of a Hausdroff space is closed.
  \end{prop}

  \begin{prop}
    Any closed subset of a compact space is compact.
  \end{prop}

  \begin{prop}
    Let $f:X\to Y$ be injective and continuous. Then if $Y$ is Hausdorff, $X$ is Hausdorff.
  \end{prop}

  \begin{prop}
    Let $X$ be a compact topological space, and $Y$ a Hausdorff space. 
    Let $f:X\to Y$ be surjective and continuous. Then $f$ is a quotient map.
  \end{prop}

  \begin{proof}
    Let $U\subset Y$ be non-empty. Since $f$ is surjective, $f^{-1}(U)$ is non-empty. 
    Suppose $f^{-1}(U)$ is open in $X$. Then since $X\backslash f^{-1}(U)$ is closed and $X$ is compact, 
    $X\backslash f^{-1}(U)$ is compact. Then since $f$ is continuous, $f(X\backslash f^{-1}(U))$ 
    is compact.\\
    
    Then since $Y$ is Hausdorff, $f(X\backslash f^{-1}(U))=Y\backslash f(f^{-1}(U))=Y\backslash U$ is closed, 
    and hence $U$ is open, so $f$ is a quotient map.
  \end{proof}

\subsection{Construction of topological spaces}
\label{sec:2.5}

  \begin{definition}[Disjoint union]
    Consider a family of non-empty sets $X_i$ indexed by $I$. Let $X_i^*=\{(x,i):x\in X_i\}.$
    Then we define the \textit{disjoint union} of the $X_i$ as 
    $$\coprod_{i\in I}{X_i}=\bigcup_{i\in I}{X_i^*}.$$
  \end{definition}

  \begin{definition}[Disjoint union topology]
    Let $X=\coprod_{i\in I}{X_i}$. For each $i\in I$, we define 
    $$\phi_i:X_i\to X:x\mapsto(x,i)$$
    to be the \textit{canonical injection}. Then we say $U$ is open in $X$ if and only if 
    $\phi_i^{-1}(U)$ is open in $X_i$, $\forall i\in I$.\\

    Equivalently, $U$ is open in $X$ if and only if $U\cap X_i$ is open in $X_i$, $\forall i\in I$.\\

    Note that this topology is the finest topology on $X$ such that all of the canonical injections 
    $\phi_i$ are continuous. In other words, it is the final topology on $X$ with respect to the $\phi_i$.
  \end{definition}

  We use this to construct some topological spaces.

  \begin{enumerate}
    \item Let $A$ be a subset of a topological space $X$, and $q:A\to Y$ be a quotient map.
          We define an equivalence relation $\sim$ on $A$ by
          $$x\sim y \Leftrightarrow f(x)=f(y),\quad\forall x,y\in A.$$
          Then we define $X\cup_f Y=X/\sim$.\\

          If $Y$ is a singlet set, then we call this \textit{collapsing}, and 
          denote it $\frac{X}{A}$.
    \item Instead let $f:A\to Y$ be continuous, and define $\sim$ on $Y$ by 
          $x\sim f(x)\quad\forall x\in A$. Then we define 
          $$X\cup_f Y=X\sqcup Y/\sim.$$
  \end{enumerate}

\subsection{Lie Groups}
\label{sec:2.6}

  \begin{definition}[Manifold]
    An \textit{$n$-dimensional manifold} is a topological space $X$ with is locally homeomorphic to $\R^n$. 
    That is, $\forall x\in X$, there exists an open neighbourhood $U\subset X$ of $x$ such that there is a function 
    $$f:U\to\R^n$$
    which is a homeomorphism. 
  \end{definition}

  \begin{definition}[Lie group]
    A \textit{Lie group} is a topological group in which the group operations are smooth (inifinitely differentiable), 
    and which is also a smooth manifold.
  \end{definition}

