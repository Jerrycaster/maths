\documentclass[../summer.tex]{subfile}

\subsection{Initial and Final Topologies}
\label{sec:2.1}

  \begin{definition}[Initial topology]
    Given a family of topological spaces $Y_i$ indexed by $i\in I$, the \textit{initial topology} 
    with respect to the functions
    $$f_i:X\to Y_i$$
    is the coursest topology that can be given to $X$ such that those functions are continuous.\\

    Explicitly, it is the topology generated by taking all finite intersections and arbitrary unions of 
    sets of the form $f_i^{-1}(U)$, where $U$ is open in $Y_i$. Consider the following examples:

    \begin{enumerate}
      \item Let $A\subset X$. The subspace topology $\tau_A=\{A\cap U:U\text{ is open in }X\}$
            is the initial topology on $A$ with respect to the canonical inclusion map 
            $$i:A\hookrightarrow X.$$
      \item Given a family of topological spaces $X_i$ indexed by $I$, let $X=\prod_{i\in I}{X_i}.$
            the product topology on $X$ is the initial topology on $X$ with respect to 
            the canonical projections 
            $$p_i:X\to X_i\quad\forall i\in I.$$
    \end{enumerate}
  \end{definition}

  \begin{theorem}[Characteristic property of initial topologies]
    Let $X,Y,Z_i$ be topological spaces, with $Z_i$ indexed by $i\in I$ and $Y$ given the initial 
    topology with respect to the functions
    $$g_i:Y\to Z_i.$$
    Let $f:X\to Y$. Then $f$ is continuous if and only if $g_i\circ f$ is continuous $\forall i\in I$.\\

    In other words, the following diagram is commutative:
    \[ \psset{arrows=->, arrowinset=0.25, linewidth=0.6pt, nodesep=3pt, labelsep=2pt, rowsep=0.7cm, colsep = 1.1cm, shortput =tablr}
    \everypsbox{\scriptstyle}
    \begin{psmatrix}
    A & B\\%
    A_f & B_g
    %%%
    \ncline{1,1}{1,2}^{\varphi} \ncline{1,1}{2,1} <{\varrho_f }
    \ncline{1,2}{2,2} > {\varrho_g}
    \ncline{2,1}{2,2}^{\varphi_f}
    \end{psmatrix}
    \]
   
   \[ \begin{tikzcd}
   A \arrow{r}{\varphi} \arrow[swap]{d}{\varrho_f} & B \arrow{d}{\varrho_g} \\%
   A_f \arrow{r}{\varphi_f}& B_g
   \end{tikzcd}
   \]

  \end{theorem}

  \begin{definition}[Final topology]
    The \textit{final topology} on a non-empty set $X$ with respect to a 
    family of functions $f_i$ indexed by $i\in I$ is the finest topology on $X$ such that 
    the function $f_i$ are continuous.\\

    Explicitly, given a family of topological spaces $Y_i$ indexed by $i\in I$, the final topology is 
    the finest topology on $X$ with respect to the functions
    $$f_i:Y_i\to X$$
    such that they are continuous.\\

    Succinctly, a subset $U\subset X$ is an element of the final topology if and only if 
    $f_i^{-1}(U)$ is open in $Y_i$, $\forall i\in I$.
  \end{definition}

  \begin{theorem}[Universal property of final topologies]

  \end{theorem}

\subsection{Quotients}
\label{sec:2.2}

  \begin{definition}[Binary relation]
    Let $X$ be a non-empty set. A \textit{binary relation}, $\sim$, on $X$ is a map 
  $$\cdot\sim\cdot:X\times X\to\{\text{True, False}\}.$$
  \end{definition}

  \begin{definition}[Equivalence relation]
    Let $\sim$ be a binary operation on a non-empty set $X$. We say $\sim$ is an 
    \textit{equivalence relation} if it is
    \begin{enumerate}
      \item Reflexive: $x\sim x \quad\forall x\in X$
      \item Symmetric: $x\sim y$ $\Leftrightarrow$ $y\sim x\quad\forall x,y\in X$
      \item Transitive: $x\sim y$, $y\sim z$ $\Rightarrow$ $x\sim z\quad\forall x,y,z\in X$
    \end{enumerate}
  \end{definition}

  \begin{prop}
    Let $\sim$ be an equivalence operation on a non-empty set $X$. Given an $x\in X$, we define 
    the \textit{equivalence class} of $x$, $[x]$, to be the set of elements in $X$ equivalent to $x$ 
    under $\sim$. In other words:
    $$[x]=\{y\in X:y\sim x\}.$$
    Then the equivalence classes partition $X$.
  \end{prop}

  \begin{proof}

  \end{proof}


  \begin{definition}[Quotient space]
    Let $X$ be a non-empty set, and $\sim$ an equivalence relation on $X$. We define the \textit{quotient space}, $X/\sim$ to 
    be the set of equivalence classes of $X$. I.e.
    $$X/\sim\,=\{[x]:x\in X\}.$$
  \end{definition}

  \begin{definition}[Quotient topology]
    Let $(X,\tau_X)$ be a topological space. We define the \textit{natural map}, $p:X\to X/\sim$, by $p(x)=[x]$ $\forall x\in X$. 
    We define the \textit{quotient topology}, $\tau_\sim$, to be
    $$\tau_\sim = \{U\in X/\sim\,:p^{-1}(U)\in\tau_X\}.$$
    Note that $\tau_\sim$ is finest topology on $X/\sim$ such that the natural map $p$ is continuous.
    We say that it is the \textit{final topology} on $X/\sim$ with respect to $p$.
  \end{definition}

  \begin{definition}[Quotient map]
    We say $q:X\to Y$ between topological spaces $X$ and $Y$ is a \textit{quotient map} 
    if it is surjective, and $U\subset Y$ is open in $Y$ if and only if $q^{-1}(U)\subset X$ 
    is open in $X$.
  \end{definition}

  \begin{example}
    The natural map $p:X\to X/\sim$ is a quotient map.
  \end{example}

  \begin{proof}
    $p$ is surjective, since $\forall[x]\in X/\sim$, $p(x)=[x]$. The second property follows from the 
    definition of $p$.
  \end{proof}

  \begin{prop}
    Let $q:X\to Y$ be a quotient map. Then $f:Y\to Z$ is continuous if and only if 
    $f\circ q:X\to Z$ is continuous.
  \end{prop}

  \begin{proof}
    If $f$ is continuous, then $f\circ q$ is continuous by composition. Now assume 
    $f\circ q$ to be continuous, and let $U\subset Z$ be open in $Z$. Then 
    $(f\circ q)^{-1}(U)=q^{-1}(f^{-1}(U))$ is open in $X$, so $f^{-1}(U)$ is open in $Y$.
  \end{proof}

  \begin{prop}[Universal property of quotients]
    If $f:X\to Y$ is continuous and constant on equivalence classes of $X$. I.e. 
    $f|_{[x]}$ is constant $\forall[x]\in X/\sim.$\\

    Then there exists a unique continuous map $g:X/\sim\,\to Y$ such that 
    $$f=g\circ p.$$
    We say that $f$ descends to the quotient.
  \end{prop}

  \begin{proof}
    So given $x_i\in X$, $f(x)=c_i$ $\forall x\in[x_i]$. Then we define $g:X/\sim\,\to Y$ by 
    $g([x_i])=c_i$. Clearly this choice of $g$ is unique. We first check this is well-defined:\\

    Suppose $[x_i]=[y_i]$. Then if $y\in[y_i]$, $y\in[x_i]$, so $g([y_i])=c_i$. Now, 
    $g\circ p(x_i)=g(p(x_i))=g([x_i])=c_i=f(x_i)$. Hence $g\circ p=f$.\\

    We now check $g$ is continuous. Since $p$ is a quotient map and $g\circ p$ is continuous, 
    $g$ is continuous by the previous proposition.
  \end{proof}

\subsection{Review of point-set topology}
\label{sec:2.3}

  \begin{prop}
    Any compact subset of a Hausdroff space is closed.
  \end{prop}

  \begin{prop}
    Any closed subset of a compact space is compact.
  \end{prop}

  \begin{prop}
    Let $f:X\to Y$ be injective and continuous. Then if $Y$ is Hausdorff, $X$ is Hausdorff.
  \end{prop}

  \begin{prop}
    Let $X$ be a compact topological space, and $Y$ a Hausdorff space. 
    Let $f:X\to Y$ be surjective and continuous. Then $f$ is a quotient map.
  \end{prop}

  \begin{proof}
    Let $U\subset Y$ be non-empty. Since $f$ is surjective, $f^{-1}(U)$ is non-empty. 
    Suppose $f^{-1}(U)$ is open in $X$. Then since $X\backslash f^{-1}(U)$ is closed and $X$ is compact, 
    $X\backslash f^{-1}(U)$ is compact. Then since $f$ is continuous, $f(X\backslash f^{-1}(U))$ 
    is compact.\\
    
    Then since $Y$ is Hausdorff, $f(X\backslash f^{-1}(U))=Y\backslash f(f^{-1}(U))=Y\backslash U$ is closed, 
    and hence $U$ is open, so $f$ is a quotient map.
  \end{proof}

\subsection{Construction of topological spaces}
\label{sec:2.4}

  \begin{definition}[Disjoint union]
    Consider a family of non-empty sets $X_i$ indexed by $I$. Let $X_i^*=\{(x,i):x\in X_i\}.$
    Then we define the \textit{disjoint union} of the $X_i$ as 
    $$\bigsqcup_{i\in I}{X_i}=\bigcup_{i\in I}{X_i^*}.$$
  \end{definition}

  \begin{definition}[Disjoint union topology]
    Let $X=\coprod{i\in I}{X_i}$. For each $i\in I$, we define 
    $$\phi_i:X_i\to X:x\mapsto(x,i)$$
    to be the \textit{canonical injection}. Then we say $U$ is open in $X$ if and only if 
    $\phi_i^{-1}(U)$ is open in $X_i$, $\forall i\in I$.\\

    Equivalently, $U$ is open in $X$ if and only if $U\cap X_i$ is open in $X_i$, $\forall i\in I$.\\

    Note that this topology is the finest topology on $X$ such that all of the canonical injections 
    $\phi_i$ are continuous. In other words, it is the final topology on $X$ with respect to the $\phi_i$.
  \end{definition}

  We use this to construct some topological spaces.

  \begin{enumerate}
    \item Let $A$ be a subset of a topological space $X$, and $q:A\to Y$ be a quotient map.
          We define an equivalence relation $\sim$ on $A$ by
          $$x\sim y \Leftrightarrow f(x)=f(y),\quad\forall x,y\in A.$$
          Then we define $X\cup_f Y=X/\sim$.\\

          If $Y$ is a singlet set, then we call this \textit{collapsing}, and 
          denote it $\frac{X}{A}$.
    \item Instead let $f:A\to Y$ be continuous, and define $\sim$ on $Y$ by 
          $x\sim f(x)\quad\forall x\in A$. Then we define 
          $$X\cup_f Y=X\sqcup Y/\sim.$$
  \end{enumerate}
