\documentclass[../research.tex]{subfile}

\subsection{Quotients}
\label{sec:2.1}

  \begin{definition}[Binary relation]
    Let $X$ be a non-empty set. A \textit{binary relation}, $\sim$, on $X$ is a map 
  $$\cdot\sim\cdot:X\times X\to\{\verb%True, False%\}.$$
  \end{definition}

  \begin{definition}[Equivalence relation]
    Let $\sim$ be a binary operation on a non-empty set $X$. We say $\sim$ is an 
    \textit{equivalence relation} if it is
    \begin{enumerate}
      \item Reflexive: $x\sim x \quad\forall x\in X$
      \item Symmetric: $x\sim y$ $\Leftrightarrow$ $y\sim x\quad\forall x,y\in X$
      \item Transitive: $x\sim y$, $y\sim z$ $\Rightarrow$ $x\sim z\quad\forall x,y,z\in X$
    \end{enumerate}
  \end{definition}

  \begin{prop}
    Let $\sim$ be an equivalence operation on a non-empty set $X$. Given an $x\in X$, we define 
    the \textit{equivalence class} of $x$, $[x]$, to be the set of elements in $X$ equivalent to $x$ 
    under $\sim$. In other words:
    $$[x]=\{y\in X:y\sim x\}.$$
    Then the equivalence classes partition $X$.
  \end{prop}

  \begin{proof}
    Since every element $x_i\in X$, $i\in I$, is equivalent to itself, $x_i\in[x_i]$, so 
    $$X = \bigcup_{i\in I}{[x_i]}.$$
    So equivalence classes cover $X$.\\

    Then if $x\nsim y$, $y\nsim x$, so $y\notin[x]$. And if $z\in[y]$, $z\sim y$ so $z\nsim x$, 
    and so $z\notin [x]$. Hence
    $$x\nsim y \Leftrightarrow [x]\cap[y]=\varnothing.$$
    So equivalence classes partition $X$.
  \end{proof}

  \begin{definition}[Quotient set]
    Let $X$ be a non-empty set, and $\sim$ an equivalence relation on $X$. We define the \textit{quotient set}, 
    or \textit{quotient space}, $\X$ to be the set of equivalence classes of $X$. I.e.
    $$\X=\{[x]:x\in X\}.$$
    When $X$ has an equivalence relation, $\sim\,$, we call $(X,\sim)$ a \textit{setoid}. Another way 
    of thinking about quotient sets is by turning $\sim$ into equality, and thus turning 
    the setoid $(X,\sim)$ into a quotient set $\X$.
  \end{definition}

  \begin{example}[Quotient group]
    Let $(G,+)$ be a group, and $N$ a normal subgroup of $G$, where we write $xy$ for $x+y$ $\forall x,y\in G$. We define an equivalence relation 
    $\sim$ on $G$ to be such that 
    $$x\sim y\text{ if and only if }Nx=Ny.$$    
    Then equivalence classes $[x]$ are the cosets of $N$ in $G$, $Nx=\{nx:n\in N\}$, so the quotient set
    $$\G=\{Nx:x\in G\}.$$
    Equivalently, two elements of $G$ are equivalent if they lie in the same coset. 
    Then we define the \textit{quotient group} 
    $$\frac{G}{N}=(\G,+)$$
    where the binary operation $\cdot+\cdot:\frac{G}{N}\times\frac{G}{N}\to\frac{G}{N}$ is defined by 
    $$(Nx)(Ny)=N(xy).$$
  \end{example}

  \begin{prop}
    $\frac{G}{N}$ is indeed a group under $+$.
  \end{prop}

  \begin{proof}
    We first check that $+$ is well-defined on $\frac{G}{N}$. \\
    
    Suppose that $Nx=Nx'$, and $Ny=Ny'$. 
    So $xx'^{-1}\in N$ and $yy'^{-1}\in N$. Then $xx'^{-1}yy'^{-1}\in N$ this implies that $xy(x'y')^{-1}\in N$, 
    since $N$ is a normal subgroup.\\

    Then $Nxy=Nx'y'$, which implies that $(Nx)(Ny)=(Nx')(Ny')$.\\

    So $+$ is well-defined. We now check the group axioms:
    \begin{enumerate}
      \item \textit{Closure}: Since $G$ is closed under $+$, $\frac{G}{N}$ is clearly closed under $+$.
      \item \textit{Associativity}: \begin{align*}
                                      \forall x,y,z\in G, (Nx)((Ny)(Nz)) &= (Nx)(N(yz)) \\
                                                                              &= Nx(yz) \\
                                                                              &= N(xy)z \\
                                                                              &= (Nxy)(Nz) \\
                                                                              &= ((Nx)(Ny))(Nz)
                                    \end{align*}
      \item \textit{Identity}: $N$ is the identity. To see this, observe that $N(Nx)=Nx=(Nx)N$.
      \item \textit{Inverses}: Given $Nx$, $(Nx)^{-1}=Nx^{-1}$. To see this, observe that $(Nx)(Nx^{-1})=N=(Nx^{-1})(Nx)$.
    \end{enumerate}
  \end{proof}

  \begin{example}[Quotient ring]
    Let $(R,+,\times)$ be a ring, and $I$ an ideal of $R$. We can define an equivalence relation, 
    $\sim$, on $R$ by 
    $$r\sim s\Longleftrightarrow I+r=I+s.$$
    Then $(R/\sim,+,\times)$ forms a ring under $+$ and $\times$, called the \textit{quotient ring}, 
    $R/I$. \\

    We can construct the field of complex numbers, $\mathbb{C}$, by taking the following quotient: \\

    Consider $I:=(x^2+1)\R[x]$ as an ideal of $\R[x]$. We define an equivalence relation $\sim$ on $\R[x]$ 
    by 
    $$f(x)\sim g(x)\Longleftrightarrow I+f(x)=I+g(x)\quad,\forall f(x),g(x)\in\R[x].$$
    I claim that the quotient ring $\R[x]\,/\sim$,
    $$\frac{\R[x]}{I}=\{a(I+x)+(I+b):a,b\in\R\}$$.
    \begin{proof}
      Fix a coset $I+p(x)$, $p(x)\in\R[x]$. Since $\R$ is a field, $\R[x]$ is a Euclidean domain, so 
      $\exists q(x),r(x)\in\R[x]$, with $deg(r(x))<deg(x^2+1)=2$, or else $r(x)=0$, such that 
      $$p(x)=q(x)(x^2+1)+r(x).$$
      Then we write the quotient ring as 
      \begin{align*}
        \frac{\R[x]}{I} &=\{I+p(x):p(x)\in\R[x]\} \\
                        &=\{I+q(x)(x^2+1)+r(x):deg(r)<2\} \\
                        &=\{I+r(x):deg(r)<2\} \\
                        &=\{I+(ax+b):a,b\in\R\} \\
                        &=\{a(I+x)+(I+b):a,b\in\R\}
      \end{align*}
    \end{proof}
    Note that $(I+x)^2=I+x^2=(I+x^2)+(I+1)-(I+1)=-(I+1)$. This is clearly isomorphic to $\mathbb{C}$, 
    since $i^2=-1$ when we identify $i$ with $I+x$.
  \end{example}

  \begin{example}[Quotient topology]
    Let $(X,\tau_x)$ be a topological space, $Y$ a set, and $f:X\to Y$ a surjective function. Then a subset 
    $U\subset Y$ is open in $Y$ if and only if $f^{-1}(U)$ is open in $X$. In other words, this 
    topology, called the \textit{quotient topology}, is the finest topology we can define on $Y$ such 
    that $f$ is continuous, called the \textit{final topology} on $Y$ with respect to $f$. \\

    The key example here is given by defining an equivalence relation, $\sim$ on $X$, and then 
    taking $Y=\X$. We then define the \textit{natural map}, $p:X\to\X$, by $p(x)=[x]$ $\forall x\in X$. \\

    Then $(\X,\tau)$ is a topological space, where the quotient topology, $\tau$, is defined by
    $$\tau=\{U\in \X:p^{-1}(U)\in\tau_X\}.$$
    Note that $\tau$ is the final topology on $\X$ with respect to $p$.
  \end{example}

  \begin{definition}[Separateness conditions]
    \verb%%
    \begin{enumerate}
      \item Hausdorff: A topological space $X$ is called \textit{Hausdorff} iff $\forall x,y\in X,\, 
            \exists U_x,U_y\text{ such that }U_x\cap U_y=\varnothing$, where $U_x,U_y$ are open 
            neighbourhoods of $x,y$ respectively. \\

            This asserts that every two points in $X$ can be `separated' by open sets - open 
            neighbourhoods that are disjoint. Conversely, $X$ is non-Hausdorff if there exists a pair of points 
            $x,y\in X$ such for every open neighbourhood $U_x$ of $x$ and $U_y$ of $y$, $U_x\cap U_y
            \neq\varnothing$. \\

            For example, if $X=\{x,y,z\}$ and $\tau=\{\varnothing,\{x\},\{x,y\},\{x,z\},X\}$, then $X$ is 
            non-Hausdorff since the open neighbourhoods of $x$ are $\{x\},\{x,y\},\{x,z\}$, and the only open 
            neighbourhoods of $y,z$  $\{x,y\},\{x,z\}$ respectively, which are all clearly non-disjoint.
      \item Regular: A topological space $X$ is called \textit{regular} iff for any $x\in X$ and any 
            closed set $V\subset X$, there exists an open neighbourhood $U_x$ of $x$ and an open 
            superset $U_V$ of $V$, such that $U_x\cap U_V=\varnothing$.
      \item Normal: A topological space $X$ is called \textit{normal} iff for any pair of closed sets 
            $V_1,V_2\subset X$ there exist open supersets $U_1,U_2$ of $V_1,V_2$ such that $U_1\cap U_2=\varnothing$.
    \end{enumerate}
  \end{definition} 

  \begin{example}
    Let $X$ be a non-Hausdroff topological space. Then we can turn $X$ into a Hausdorff space by 
    considering the equivalence relation $\sim$ on $X$ defined by 
    $$x\sim y\Longleftrightarrow U_x\cap U_y\neq\varnothing$$
    for all open neighbourhoods $U_x,U_y$ of $x,y$ respectively. \\

    Then the quotient space $\X$ is Hausdorff when given the quotient topology.
  \end{example}

  \begin{proof}
    So $U\subset\X$ is open if and only if $p^{-1}(U)$ is open, where $p:X\to\X$ maps $x\in X$ to 
    the set of points which cannot be separated from $x$ by open sets, denoted $[x]$. \\

    So we fix distinct $[x],[y]\in\X$. Now, $p^{-1}(\{[x]\})$ is 
    open in $X$, so $\{[x]\}$ is open in $\X$. Similarly, $\{[y]\}$ is open in $\X$. Finally, note 
    that $\{[x]\}\cap\{[y]\}=\varnothing$, and so $\X$ is Hausdorff.
  \end{proof}

  \begin{definition}[Quotient map]
    We say $q:X\to Y$ between topological spaces $X$ and $Y$ is a \textit{quotient map} 
    if it is surjective, and $U\subset Y$ is open in $Y$ if and only if $q^{-1}(U)\subset X$ 
    is open in $X$.
  \end{definition}

  \begin{example}
    The natural map $p:X\to \X$ is a quotient map.
  \end{example}

  \begin{proof}
    $p$ is surjective, since $\forall[x]\in \X$, $p(x)=[x]$. The second property follows from the 
    definition of $p$.
  \end{proof}

  \begin{prop}
    Let $q:X\to Y$ be a quotient map. Then $f:Y\to Z$ is continuous if and only if 
    $f\circ q:X\to Z$ is continuous.
  \end{prop}

  \begin{proof}
    If $f$ is continuous, then $f\circ q$ is continuous by composition. Now assume 
    $f\circ q$ to be continuous, and let $U\subset Z$ be open in $Z$. Then 
    $(f\circ q)^{-1}(U)=q^{-1}(f^{-1}(U))$ is open in $X$, so $f^{-1}(U)$ is open in $Y$.
  \end{proof}

  \begin{prop}[Universal property of quotients]
    If $f:X\to Y$ is continuous and constant on equivalence classes of $X$. I.e. 
    $f|_{[x]}$ is constant $\forall[x]\in\X.$\\

    Then there exists a unique continuous map $g:\X\to Y$ such that 
    $$f=g\circ p.$$
    We say that $f$ descends to the quotient.
  \end{prop}

  \begin{proof}
    So given $x_i\in X$, $f(x)=c_i$ $\forall x\in[x_i]$. Then we define $g:\X\to Y$ by 
    $g([x_i])=c_i$. Clearly this choice of $g$ is unique. We first check this is well-defined:\\

    Suppose $[x_i]=[y_i]$. Then if $y\in[y_i]$, $y\in[x_i]$, so $g([y_i])=c_i$. Now, 
    $g\circ p(x_i)=g(p(x_i))=g([x_i])=c_i=f(x_i)$. Hence $g\circ p=f$.\\

    We now check $g$ is continuous. Since $p$ is a quotient map and $g\circ p$ is continuous, 
    $g$ is continuous by the previous proposition.
  \end{proof}

\subsection{Group Actions}
\label{sec:2.2}

  \begin{definition}[Automorphism group]
    Given a mathematical structure $X$, an \textit{automorphism} is an isomorphism, $\phi$, from $X$ to itself. Explicitly,
    $$\phi:X\to X.$$
    Coloquially, automorphisms are invertible mappings that preserve structure. In fact, they form a group under composition, 
    called the \textit{automorphism group} of $X$, which we denote $\Aut{X}$.
  \end{definition}

  \begin{example}
    \verb%%
    \begin{enumerate}
      \item If $X$ is a set with $n$ elements, then
            $$\Aut{X}=S_n,$$
            the group of permutations of the elements of $X$.
      \item If $V$ is an $n$-dimensional vector space over a field $F$, then 
            $$\Aut{V}=\text{GL}_n(F),$$
            the group of invertible $n\times n$ matrices over $F$, or equivalently the group of invertible linear 
            transformations from $V$ to itself.
      \item If $(X,\tau)$ is a topological space, then 
            $$\Aut{X}=\Homeo{X},$$
            the group of homeomorphisms of $X$. Note that homeomorphism groups of homeomorphic spaces 
            are isomorphic as groups. \\
    \end{enumerate}
  \end{example}

  Given a set $X$, we can think of $\Aut{X}$ as \textit{acting} on $X$ in an invertible way.
  Explicitly, we describe the \textit{group action} of $\Aut{X}$ on $X$ by the function:
  $$\Aut{X}\times X\to X:(\sigma,x)\mapsto \sigma\cdot x,\quad\forall\sigma\in\Aut{x},$$
  such that the following hold $\forall x\in X$, where $\sigma\cdot x$ refers to outcome of acting 
  on $x\in X$ by an automorphism $\sigma\in\Aut{X}$, and the identity in $\Aut{X}$ is denoted $e$:
  \begin{enumerate}
    \item \textit{Identity}: $e\cdot x=x$
    \item \textit{Compatibility}: $(\sigma\rho)\cdot x=\sigma\cdot(\rho\cdot x),\quad\forall
                                    \sigma,\rho\in\Aut{X}$
  \end{enumerate}

  In fact, this is a special case of an arbitrary group $G$ acting on $X$. We specify the action 
  by describing a homomorphism
  $$\varphi:G\to\Aut{X}.$$
  Which amounts to assigning an automorphism of $X$ to each element of $G$ in such a way that
  \begin{enumerate}
    \item The identity automorphism in $\Aut{X}$ is assigned to the identity in $G$,
    \item The composition of the two automorphisms assigned to two elements $g,h\in G$ is assigned to 
          the product $gh$.
  \end{enumerate}
  Since $\varphi(G)$ is a subgroup of $\Aut{X}$, we define the action of $G$ on $X$ by 
  the group action of the homomorphic image of $G$, in $\Aut{X}$, on $X$. So $g\cdot x=\varphi(g)x$, 
  where $\varphi(g)x$ is the result of applying the automorphism $\varphi(g)\in\Aut{X}$ to $x\in X$. 
  Specifically:

  \begin{definition}[Group action]
    Let $G$ be a group and $X$ a set. The \textit{(left) group action}, is a function
    $$G\times X\to X:(g,x)\mapsto g\cdot x,$$
    such that the following hold $\forall x\in X$, where we $e$ is the identity in $G$:
    \begin{enumerate}
      \item \textit{Identity}: $e\cdot x=x$
      \item \textit{Compatibility}: $(gh)\cdot x=g\cdot(h\cdot x)\quad\forall g,h\in G$
    \end{enumerate}
    Here $G$ is called a \textit{transformation group}, $X$ is called a \textit{$G$-set}.
  \end{definition}

  \begin{definition}[Orbit]
    Consider a group $G$ acting on a set $X$. We define the \textit{orbit}, $G\cdot x$,
    of $x\in X$ to be the set of elements in $X$ to which $x$ can be mapped to by the elements of $G$. Explicitly:
    $$G\cdot x=\{g\cdot x:g\in G\}.$$
  \end{definition}

  We can define an equivalence relation $\sim$ on $X$ by
  $$x\sim y\text{ if and only if }G\cdot x=G\cdot y,\quad x,y\in X.$$
  Equivalently, $x\sim y$ if and only if $\exists g\in G$ such that $g\cdot x=y$ (and so $g^{-1}\cdot y=x$). 
  Then the orbits of $X$, under the action of $G$, partition $X$. The quotient set induced by $\sim$ is
  $$\X=\{G\cdot x:x\in X\}.$$
  This is the set of all orbits of $X$ under the action of $G$, called the \textit{quotient} of the action and denoted $\frac{X}{G}$.

  \begin{definition}[Stabiliser subgroup]
    Given $g\in G$ and $x\in X$ such that $g\cdot x=x$, we say that $g$ fixes $x$, in other words 
    $g$ acts as the identity on $x$. So for all $x\in X$, we define the \textit{stabiliser subgroup}, 
    $G_x\subset G$, of $G$ with respect to $x$ as the set of elements in $G$ that fix $x$. Explicitly, given $x$,
    $$G_x=\{g\in G:g\cdot x=x\}.$$
  \end{definition}

  \begin{prop}
    $G_x$ is indeed a subgroup of $G$.
  \end{prop}

%  \begin{prop}
%    Let $\psi:G\to\text{Aut}(X)$ be a homomorphism. Then
%    $$\text{Ker}(\psi)=\bigcap_{x\in X}{G_x}.$$
%  \end{prop}

  $G_x$ is generally not a normal subgroup of $G$. However, given an $x\in X$, we can define an 
  equivalent relation, $\sim$, on $G$ by 
  $$g\sim h\Longleftrightarrow G_xg=G_xh,\quad g,h\in G.$$
  We can then form the quotient set $\G$ to be the set of cosets of $G_x$ in $G$, denoted $G/G_x$. 
  Explicitly, given $x\in X$,
  $$\frac{G}{G_x}=\{G_xg:g\in G\}.$$

  \begin{lemma}
    Given $x,y\in X$, and $g\in G$ such that $y=g\cdot x$, 
    the two stabiliser groups $G_x,G_y$ are related by 
    $$G_y=g^{-1}G_x g.$$
  \end{lemma}

  \begin{proof}
    Fix $x,y\in X$, and let $g\in G$ be such that $y=g\cdot x$, and fix $h\in G_y$. Then 
    \begin{align*}
      h\cdot y= y &\Longleftrightarrow h\cdot(g\cdot x)=g\cdot x \\
                  &\Longleftrightarrow (hg)\cdot x=g\cdot x \\
                  &\Longleftrightarrow g^{-1}\cdot((hg)\cdot x)=g^{-1}\cdot(g\cdot x) \\
                  &\Longleftrightarrow (g^{-1}hg)\cdot x=(g^{-1}g)\cdot x = x \\
                  &\Longleftrightarrow g^{-1}hg\in G_x \\
                  &\Longleftrightarrow h\in g^{-1}G_xg
    \end{align*}
  \end{proof}

  \begin{theorem}[Orbit-stabiliser theorem]
    For a fixed $x\in X$, there exists a bijection $\psi:G/G_x\to G\cdot x.$
  \end{theorem}

  \begin{proof}
    We define a map $\psi:G/G_x\to G\cdot x$ by $\psi(G_xg)=g\cdot x$, $\forall g\in G$.
    First, we show this an injective, well-defined function: Fix $x\in X$, and let $G_xg=G_xh$. Then 
    \begin{align*}
      G_xg^{-1}h=G_x &\Longleftrightarrow g^{-1}h\in G_x \\
                     &\Longleftrightarrow (g^{-1}h)\cdot x=g^{-1}\cdot(h\cdot x)=x \\
                     &\Longleftrightarrow h\cdot x=g\cdot x \\
                     &\Longleftrightarrow \psi(G_xg)=\psi(G_xh)
    \end{align*}
    Now we show that $\psi$ is surjective: Fix $x\in X$, and $g\cdot x\in G\cdot x$.
    $$G_xg=\{hg:h\cdot x=x, \,\forall h\in G\}$$
  \end{proof}

  \begin{cor}[Burnside's lemma]
    The number of orbits of $G$ in $X$, $|X/G|$, is given by
    $$|X/G|=\frac{1}{|G|}\sum_{g\in G}{|X^g|},$$
    where $X^g\subset X$ is the set of elements in $X$ fixed by $g$.
  \end{cor}  
  
  In fact, if $G$ is finite, then by Lagrange's theorem,
  $$|G|=|G\cdot x||G_x|.$$

  \begin{definition}[Topological group]
    A \textit{topological group} is a non-empty set $G$ which is both a group and topological space, 
    in which the group operations of product:
    $$G\times G\to G:(x,y)\mapsto xy,\quad x,y\in G$$
    and taking inverses:
    $$G\to G:x\mapsto x^{-1},\quad x\in G$$
    are continuous, where $G\times G$ is given the product topology.

    For example, we can turn every finite group into a topological group by giving it the discrete 
    topology. Returning to our idea of group actions, since the automorphism groups of a topological 
    space $X$ is $\Homeo{X}$, we can think of $\Homeo{X}$ as acting of $X$

    Let $X$ be a topological space, and $G$ a topological group. Then we say the action of $G$ on 
    $X$, 
    $$\phi:G\times X\to X:(g,x)\mapsto g\cdot x,$$
    is \textit{continuous} if $\phi^{-1}(U)$ is open in $G\times X$, with respect to the product 
    topology, whenever $U$ is open in $X$. 

    We then call $X$ a \textit{$G$-space}. \\ 

    Another way of thinking about a continuous group action is if we define a homomorphism
    $$\varphi:G\to\Homeo{X},$$
    then the action of an element $g\in G$ on $x\in X$ is $g\cdot x:=\varphi(g)x$.

      %\begin{definition}[Homomorphism of topological groups]
  %  Let $G,H$ be topological groups. Then a \textit{homomorphism of topological groups} is a 
  %  structure-preserving map (i.e. a group homomorphism), $\phi:G\to H$, which is continuous. Explicitly,
  %  \begin{enumerate}
  %    \item $\phi(xy)=\phi(x)\phi(y),\quad x,y\in G$
  %    \item $\phi^{-1}(U)$ is open in $G$ whenever $U$ is an open subset of $H$.
  %  \end{enumerate}
%
  %  An \textit{isomorphism} is a bijective group homomorphism which is also a homeomorphism of the 
  %  underlying topological spaces.
  %\end{definition}
  \end{definition}

\subsection{Initial and Final Topologies}
\label{sec:2.3}

  Given a set $X$, the indiscrete topology is the coursest topology we can give to $X$. 
  However, for a family of topological spaces $Y_i$ indexed by $i\in I$, the functions 
  $$f_i:X\to Y_i$$
  will almost never be continuous. The initial topology is the coursest topology such that 
  they are.

  \begin{definition}[Initial topology]
    Given a family of topological spaces $Y_i$ indexed by $i\in I$, the \textit{initial topology} 
    with respect to the functions
    $$f_i:X\to Y_i$$
    is that in which open sets $U\subset X$ are those for which $f_i(U)$ is open in $Y_i$ $\forall i\in I$. 
    If $X$ already has a topology, then these functions are continuous if and only if the existing 
    topology is finer than the initial topology. Thus, the initial topology is the coursest topology that 
    can be defined on $X$ such that these functions are continuous. \\

    Explicitly, it is the topology generated by taking all finite intersections and arbitrary unions of 
    sets of the form $f_i^{-1}(U_i)$, where $U_i$ is open in $Y_i$. Consider the following examples:

    \begin{enumerate}
      \item Let $A\subset X$. The subspace topology $\tau_A=\{A\cap U:U\text{ is open in }X\}$
            is the initial topology on $A$ with respect to the canonical inclusion map 
            $$i:A\hookrightarrow X.$$
      \item Given a family of topological spaces $X_i$ indexed by $I$, let $X=\prod_{i\in I}{X_i}.$
            the product topology on $X$ is the initial topology on $X$ with respect to 
            the canonical projections 
            $$p_i:X\to X_i\quad\forall i\in I.$$
    \end{enumerate}

    If $f$ is injective, the initial topology can be identified with the \textit{subspace} topology of 
    $X$, when $X$ is viewed as a subset of $Y_i$.
  \end{definition}

  \begin{theorem}[Characteristic property of initial topologies]
    Let $X,Y,Z_i$ be topological spaces, with $Z_i$ indexed by $i\in I$ and $Y$ given the initial 
    topology with respect to the functions
    $$g_i:Y\to Z_i.$$

    Let $f:X\to Y$. Then $f$ is continuous if and only if $g_i\circ f$ is continuous $\forall i\in I$. \\

    In other words, the following diagram is commutative for all $i\in I$: \\
    \begin{center}
      \begin{tikzcd}
        X \arrow{r}{f} \arrow[swap]{dr}{g_i\circ f} & Y \arrow{d}{g_i} \\
         & Z_i
      \end{tikzcd}
    \end{center}
  \end{theorem}
  
  \begin{proof}
    Assume first that $f$ is continuous. Then since $g_i$ is continuous $\forall i\in I$, $g_i\circ f$ is continuous. \\

    Now assume $g_i\circ f$ is continuous $\forall i\in I$. Let $U\subset Y$ be open in $Y$. Then $U$ is a finite intersection 
    and arbitrary union of open sets of the form $g_i^{-1}(V_i)$, where each $V_i$ is open in $Z_i$. \\

    Then $f^{-1}(U)$ is a finite intersection and arbitrary union of open sets of the form $f^{-1}(g_i^{-1}(V_i))=
    (g_i\circ f)^{-1}(V_i)$. This is equal to $(g_i\circ f)^{-1}(W_i)$, where $W_i$ is a finite intersection and arbitrary union 
    of the open sets $V_i$. \\

    Then since each $W_i$ is open, and each $g_i\circ f$ is continuous, $(g_i\circ f)^{-1}(W_i)=f^{-1}(g_i^{-1}(W_i))$ is open in $X$.
    $g_i^{-1}(W_i)$ is a finite intersection and arbitrary union of $g_i^{-1}(V_i)$, which is open in $Y$. Hence $f^{-1}(U)$ is open in $X$.
  \end{proof}

  \begin{definition}[Final topology]
    Given a family of topological spaces $X_i$ indexed by $i\in I$, the \textit{final topology} 
    with respect to the functions
    $$f_i:X_i\to Y$$
    is that in which open sets $A\subset Y$ are those for which $f_i^{-1}(U)$ is open in $X_i$ $\forall
    i\in I$. If $Y$ already has a topology, the $f_i$ are continuous precisely when the existing 
    topology is \textit{courser} than the final topology on $Y$. Thus, the final topology is the finest 
    topology that can be given to $Y$ such that these functions are continuous. \\

    If each $f_i$ is surjective, we call this topology the \textit{quotient topology} under the equivalence 
    relation $\sim$ defined by
    $$x\sim y\Longleftrightarrow f_i^{-1}(\{x\})=f_i^{-1}(\{y\})\quad\forall x,y\in X_i,\,i\in I.$$
  \end{definition}

  \begin{theorem}[Universal property of final topologies]
    Let $X_i,Y,Z$ be topological spaces, with $X_i$ indexed by $i\in I$ and $Y$ given the final 
    topology with respect to the functions
    $$f_i:X_i\to Y$$

    Then $g:Y\to Z$ is continuous if and only if $g\circ f_i$ is continuous $\forall i\in I$. \\

    In other words, the following diagram is commutative $\forall i\in I$: \\
    \begin{center}
      \begin{tikzcd}
        X_i \arrow{r}{f_i} \arrow[swap]{dr}{g\circ f_i} & Y \arrow{d}{g} \\
         & Z
      \end{tikzcd}
    \end{center}
  \end{theorem}

\subsection{Review of point-set topology}
\label{sec:2.4}

  \begin{prop}
    Any compact subset of a Hausdroff space is closed.
  \end{prop}

  \begin{prop}
    Any closed subset of a compact space is compact.
  \end{prop}

  \begin{prop}
    Let $f:X\to Y$ be injective and continuous. Then if $Y$ is Hausdorff, $X$ is Hausdorff.
  \end{prop}

  \begin{prop}
    Let $X$ be a compact topological space, and $Y$ a Hausdorff space. 
    Let $f:X\to Y$ be surjective and continuous. Then $f$ is a quotient map.
  \end{prop}

  \begin{proof}
    Let $U\subset Y$ be non-empty. Since $f$ is surjective, $f^{-1}(U)$ is non-empty. 
    Suppose $f^{-1}(U)$ is open in $X$. Then since $X\backslash f^{-1}(U)$ is closed and $X$ is compact, 
    $X\backslash f^{-1}(U)$ is compact. Then since $f$ is continuous, $f(X\backslash f^{-1}(U))$ 
    is compact. \\
    
    Then since $Y$ is Hausdorff, $f(X\backslash f^{-1}(U))=Y\backslash f(f^{-1}(U))=Y\backslash U$ is closed, 
    and hence $U$ is open, so $f$ is a quotient map.
  \end{proof}

\subsection{Construction of topological spaces}
\label{sec:2.5}

  \begin{definition}[Disjoint union]
    Consider a family of non-empty sets $X_i$ indexed by $I$. Let $X_i^*=\{(x,i):x\in X_i\}.$
    Then we define the \textit{disjoint union} of the $X_i$ as 
    $$\coprod_{i\in I}{X_i}=\bigcup_{i\in I}{X_i^*}.$$
  \end{definition}

  \begin{definition}[Disjoint union topology]
    Let $X=\coprod_{i\in I}{X_i}$. For each $i\in I$, we define 
    $$\phi_i:X_i\to X:x\mapsto(x,i)$$
    to be the \textit{canonical injection}. Then we say $U$ is open in $X$ if and only if 
    $\phi_i^{-1}(U)$ is open in $X_i$, $\forall i\in I$. \\

    Equivalently, $U$ is open in $X$ if and only if $U\cap X_i$ is open in $X_i$, $\forall i\in I$. \\

    Note that this topology is the finest topology on $X$ such that all of the canonical injections 
    $\phi_i$ are continuous. In other words, it is the final topology on $X$ with respect to the $\phi_i$.
  \end{definition}

  We use this to construct some topological spaces.

  \begin{enumerate}
    \item Let $A$ be a subset of a topological space $X$, and $q:A\to Y$ be a quotient map.
          We define an equivalence relation $\sim$ on $A$ by
          $$x\sim y \Leftrightarrow q(x)=q(y),\quad\forall x,y\in A.$$
          Then we define $X\cup_q Y=\X.$ \\

          If $Y$ is a singlet set, then we call this \textit{collapsing}, and 
          denote it $X/A$.
    \item Instead let $f:A\to Y$ be continuous, and define $\sim$ on $Y$ by 
          $x\sim f(x)\quad\forall x\in A$. Then we define 
          $$X\cup_q Y=X\sqcup Y\,/\sim.$$
  \end{enumerate}

  \begin{definition}[Attaching map]
  \end{definition}

  \begin{definition}[Adjunction space]
  \end{definition}

  \begin{lemma}[Gluing lemma]
  \end{lemma}